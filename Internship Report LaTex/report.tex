\documentclass[a4paper,12pt]{report}  % Déclare un document de type "report" (idéal pour les rapports longs)
% a4paper : Définit le format du papier en A4 (210 × 297 mm)
% 12pt : Définit la taille de police principale à 12 points (standard pour les rapports)

\usepackage[utf8]{inputenc} % Permet d'écrire directement des accents et caractères spéciaux avec l'encodage UTF-8

\usepackage[T1]{fontenc}  % Définit l'encodage des polices pour une meilleure gestion des caractères
% T1 : Active un encodage qui améliore l'affichage des accents (é, à, ç...) 
%      et la césure correcte des mots en français
% Recommandé pour les documents en français

\usepackage{amsmath, amsthm, amssymb}   % amsmath :  Pour des équations et des                                         notations mathématiques avancées.
% amsthm : Pour formater des théorèmes, définitions, propositions, etc.
% amssymb : Pour ajouter des symboles mathématiques supplémentaires

% Définition d'un style pour les définitions
\newtheoremstyle{definitionstyle}  % Nom du style
{10pt}                           % Espace avant
{10pt}                           % Espace après
{\normalfont}                    % Police normale
{}                               % Retrait
{\bfseries}                      % Titre en gras
{.}                              % Ponctuation après le titre
{1em}                            % Espace après le titre (ajout d'un petit                                    espace)
{}                               % Personnalisation du titre

\theoremstyle{definitionstyle}
\newtheorem{definition}{Définition}[chapter] % Numérotation par chapitres

% Définition d'un style pour les exemples
\newtheoremstyle{examplestyle}  % Nom du style
{10pt}                        % Espace avant
{10pt}                        % Espace après
{\normalfont}                 % Police normale
{}                            % Retrait
{\itshape}                    % Titre en italique
{.}                           % Ponctuation après le titre
{1em}                         % Espace après le titre
{}                            % Personnalisation du titre

\theoremstyle{examplestyle}
\newtheorem{example}{Exemple}[chapter] % Numérotation par section

\newtheoremstyle{remarkstyle}  
{10pt}                        % Espace avant
{10pt}                        % Espace après
{\normalfont}                 % Police normale pour le texte
{}                            % Retrait
{\itshape}                    % Titre en italique
{.}                           % Ponctuation après le titre
{1em}                         % Espace après le titre
{}                            % Personnalisation du titre

\theoremstyle{remarkstyle}
\newtheorem{remark}{Remarque}[chapter] % Numérotation par chapitres

% Définition du style pour l'environnement "proposition"
\newtheoremstyle{propositionstyle}  
{10pt}                        % Espace avant
{10pt}                        % Espace après
{\normalfont}                 % Police normale pour le texte
{}                            % Retrait
{\bfseries}                    % Titre en italique
{.}                           % Ponctuation après le titre
{1em}                         % Espace après le titre
{}                            % Pas d'autres spécifications

% Définition de l'environnement proposition
\theoremstyle{propositionstyle}
\newtheorem{proposition}{Proposition}[chapter]  % Numérotation par chapitre

% Définition du style pour l'environnement "théorème"
\newtheoremstyle{theoremstyle}  
{10pt}                        % Espace avant
{10pt}                        % Espace après
{\normalfont}                 % Police normale pour le texte
{}                            % Retrait
{\bfseries}                    % Titre en italique
{.}                           % Ponctuation après le titre
{1em}                         % Espace après le titre
{}                            % Pas d'autres spécifications

% Définition de l'environnement théorème
\theoremstyle{theoremstyle}
\newtheorem{theoreme}{Théorème}[chapter]  % Numérotation par chapitre

% Définition d'un nouveau style de théorème pour les preuves
\newtheoremstyle{proofstyle}  
{10pt}                        % Espace avant
{10pt}                        % Espace après
{\normalfont}                 % Police normale pour le texte
{}                            % Pas de retrait
{\itshape}                    % Titre en italique
{.}                           % Ponctuation après le titre
{1em}                         % Espace après le titre
{}                            % Pas d'autres spécifications

% Appliquer le style à l'environnement de preuve
\theoremstyle{proofstyle}
\newtheorem{preuve}{Preuve}[chapter]  % Numérotation par chapitre

% Définition de l'environnement de preuve avec le carré à la fin
\newenvironment{preuveenv}{\begin{proof}[Preuve]}{\qed\end{proof}}

\usepackage[a4paper, left=2.5cm, right=2.5cm, top=2.5cm, bottom=2.5cm]{geometry} % Définition des marges du document (A4 avec 2.5 cm sur chaque côté)

\usepackage[backend=biber,style=authoryear]{biblatex} % Gestion de la bibliographie
\setlength{\bibitemsep}{1\baselineskip}  % Ajuste l'espacement entre les éléments de la bibliographie

\addbibresource{references.bib} % On inclue le fichier BibTeX references

\usepackage{fancyhdr}     % En-têtes et pieds de page personnalisés

\pagestyle{fancy}         % modification flexible de l'en-tête et du pied de                             page  
\fancyhf{}                % vide l'en-tête et le pied de page
\fancyhead[L]{\leftmark}  % Section ou chapitre courant
\fancyfoot[C]{\thepage}   % Numéro de page au centre

\usepackage{hyperref}
% Ajouter des liens hypertextes dans le document                                 (liens internes, externes, etc.), mettre les liens                             en surbrillance

%%%%%%%%%%%%%%%%%%%%%%%%%%%%%%%%%%%%%%%%%%%%%%%%%%%%
%%%%%%%%%%%%%%%%%%%%%%%%%%%%%%%%%%%%%%%%%%%%%%%%%%%%
%%%%%%%%%%%%%%%%%%%%%%%%%%%%%%%%%%%%%%%%%%%%%%%%%%%%
\usepackage{csquotes}
\usepackage[english,french]{babel} % Langue du document
\usepackage{graphicx}       % Inclusion d'images
\usepackage{float}
\usepackage{color}
\usepackage{comment}        % pour commenter plusieur ligne
\usepackage{enumitem}
\usepackage{titlesec}
\usepackage{calc}
\usepackage{times}
\usepackage{lettrine}
\usepackage{algorithm}
\usepackage{algpseudocode}
\usepackage{mathtools}  % À mettre dans le préambule
\usepackage{listings}
\renewcommand\qedsymbol{$\blacksquare$}

\usepackage{listings}
\usepackage{xcolor}  % pour utiliser les couleurs

\definecolor{codegreen}{rgb}{0,0.6,0}
\definecolor{codegray}{rgb}{0.5,0.5,0.5}
\definecolor{codepurple}{rgb}{0.58,0,0.82}
\definecolor{backcolour}{rgb}{0.95,0.95,0.92}

\lstdefinestyle{mystyle}{
	backgroundcolor=\color{backcolour},   
	commentstyle=\color{codegreen},
	keywordstyle=\color{magenta},
	numberstyle=\tiny\color{codegray},
	stringstyle=\color{codepurple},
	basicstyle=\footnotesize,
	breakatwhitespace=false,         
	breaklines=true,                 
	captionpos=b,                    
	keepspaces=true,                 
	numbers=left,                    
	numbersep=5pt,                  
	showspaces=false,                
	showstringspaces=false,
	showtabs=false,                  
	tabsize=2
}

\lstset{style=mystyle}

	\lstset{language=Python}
\titleformat
{\chapter} % command
[display] % shape
{\Large} % format
{%
	\rule{\widthof{CHAPITRE}}{0.6pt}\vspace{-0.3cm}\\
	\makebox[0pt][l]{CHAPITRE \Huge\textbf{\thechapter}}\vspace{-0.6cm} \\
	\rule{\widthof{CHAPITRE \Huge\textbf{\thechapter}}}{0.6pt}
} % label
{1ex} % sep
{\huge\textit } 
[
] % after-code

\begin{document}
	
	% En-tête
	\begin{titlepage}
		\begin{center}
			% Logos
			\begin{figure}[t]
				\centering
				\includegraphics[width=0.35\textwidth]{logo/logo_LIRMM.jpg} 
			\end{figure}
			\begin{figure}[t]
				\centering
				\includegraphics[width=0.2\textwidth]{logo/logo_univ_mpt.png} 
			\end{figure}
			
			% Titre
			\vspace{2cm}
			\Large{\textbf{Rapport de stage}}
			
			\vspace{0.5cm}
				
				 \Large{\textbf{Réduction de réseaux - adaptations d'idées provenant du cas polynomiale au cas entier.}} \\
			\vspace{0.5cm}
			\large{\textit{HAI001I : Stage académique}} \\
			
			% Espacement
			\vspace{2cm}
			
			% Auteurs
			\large{
				\textbf{Lucas Noirot} \\
				(\texttt{\href{mailto:lucas.noirot@etu.umontpellier.fr}{lucas.noirot@etu.umontpellier.fr}})
			}
			
			% Espacement
			\vspace{1.5cm}
			
			% Encadrant
			\large{
				\textbf{Encadrant :} \textbf{Romain Lebreton} \\
				(\texttt{\href{mailto:romain.lebreton@lirmm.fr}{romain.lebreton@lirmm.fr}})
			}
			
			% Espacement
			\vspace{2cm}
			
			% Date
			\normalsize{
				\textbf{Date :}  17 février - 26 juin 2025
			}
		\end{center}
	\end{titlepage}
	
	%\chapter*{Remerciements}
	% à rédiger à la fin
	\tableofcontents % Table des matières
	
	\renewcommand{\listalgorithmname}{Table des algorithmes}
	\listofalgorithms % Table des algorithmes
	
	
	\chapter*{Notations}
	\addcontentsline{toc}{chapter}{Notations}  % Ajoute au sommaire si nécessaire
	
	Cette section est conçue comme un lexique dans lequel sont répertorié les notations de ce mémoire.
	
	\begin{itemize}[label={}]
		\item $\mathbb{Z}$ : L'ensemble des entiers relatifs.
		\item $\mathbb{R}$ : L'ensemble des réels.
		\item $\mathcal{L}$ : Un réseau, désigné par une lettre majuscule calligraphiée.
		\item $\mathcal{L}(B)$ le réseau engendré par la matrice $B$.
		\item $A^T$ : La transposée de la matrice $A$.
		\item $\mathbb{K}$ : Un corps quelconque.
		\item $\mathbb{F}$ : Un corps fini.
		\item $\mathbb{F}[X]$ : L'anneau des polynômes univariés à coefficients dans le corps fini $F$.
	\end{itemize}
	
	
	\chapter*{Guide de Lecture}
	\addcontentsline{toc}{chapter}{Guide de Lecture}
	
		\lettrine{\textbf{D}}{ans} ce mémoire, chaque définition est suivie d'un exemple concret illustrant la notion en question, ainsi que d'un contre-exemple visant à en exposer les subtilités et les exceptions éventuelles. Cette approche permet de mieux comprendre les conditions et les limitations associées à chaque concept. L'objectif est de clarifier les différences entre les situations où une définition est applicable et celles où elle ne l'est pas, afin de renforcer la compréhension approfondie des théorèmes et des constructions présentés.
	

	
	\chapter*{Introduction}
	
	\lettrine{\textbf{L}}{'objectif} de ce stage est d'explorer les réseaux euclidiens et polynomiaux, ainsi que d'analyser les techniques de réduction de réseau. L'enjeu est d'adapter les méthodes de réduction utilisées dans le cadre des réseaux polynomiaux au cas des réseaux entiers.
	
	La réduction de réseau est un outil essentiel dans plusieurs domaines de la cryptographie, notamment dans les algorithmes de sécurité basés sur la difficulté de résoudre des problèmes liés aux réseaux. En particulier, la réduction de réseaux polynomiaux est connue de manière polynomiale, et l'adaptation de ces méthodes pour les réseaux entiers pourrait avoir des implications importantes pour la sécurité des systèmes cryptographiques. Ce stage vise donc à approfondir cette adaptation et à en étudier les applications pratiques.	
	
	\chapter{Réseaux euclidiens et polynomiaux}
	
	\lettrine{\textbf{L}}{'étude} des réseaux euclidiens en mathématiques trouve ses origines au XVIIIe siècle, lorsque Leonhard Euler a exploré les structures géométriques des points dans l'espace. Toutefois, ce n'est que dans le courant du XXe siècle que le concept de réseaux euclidiens a été intégré à la cryptographie. Dans les années 1990, des chercheurs tels qu'Ajtai, Dwork, Regev et d'autres ont introduit l'idée des réseaux euclidiens comme fondement de problèmes complexes en cryptographie, ouvrant ainsi la voie à de nouvelles constructions cryptographiques.
	
	\section{Généralités sur les réseaux}
	
	
	\subsection{Définitions et exemples}
	Nous considérons un espace euclidien, c'est-à-dire un espace vectoriel réel de dimension finie muni d'un produit scalaire, noté $\langle f, g \rangle$. Ici, nous utiliserons le produit scalaire usuel, défini par $\langle f, g \rangle = f \cdot g^T$ lequel induit la norme-2, donnée par $\|f\|_2=(f \cdot f^T)^{1/2}$.
	
	
	Dans la littérature, on trouve les trois définitions suivantes d’un réseau euclidien qui représentent le même objet.
	
	\begin{definition}[\cite{MCA}]
		Soit $n \in \mathbb{N}$ et $f_1, \cdots, f_n \in \mathbb{R}^n$.
		
		Alors $\mathcal{L} = \sum_{1 \leq i \leq n} \mathbb{Z} f_i
		$ est un réseau euclidien
	\end{definition}

	\begin{definition}[\cite{Wallet}]
		Un \textbf{réseau euclidien} est un sous-groupe discret de $\mathbb{R}^n$
	\end{definition}

	\begin{definition}[\cite{clef_unique_2}]
		Un \textbf{réseau euclidien} $(\Lambda, q)$ est un $\mathbb{Z}$-module libre $\Lambda$ de rang fini avec une forme quadratique définie positive $q$ sur $\Lambda \otimes_\mathbb{Z} \mathbb{R}$ 
	\end{definition}
	
	Soit $\mathcal{L}$ un réseau euclidien. Il existe une famille $\mathbb{Z}$-libre maximale $(b_i)_{1 \leq i \leq n}$ dans $\mathcal{L}$ tel que $\mathcal{L} = Zb_1 \oplus \cdots \oplus Zb_n$, qu'on appelle \textbf{base du réseau} $\mathcal{L}$, si on note $B$ la matrice des $(b_i)$ on notera $L(B)$ le réseau de base $B$, donc engendré par les $(b_i)$.
	
	L'entier $n$ est commun à toutes les bases de $L$ et on l'appelle rang de $L$. Lorsque $n=m$, on dit que le réseau est de rang plein.
	
	\begin{proposition}%Wallet
		Soit $L$ et $L'$ deux réseaux de rang $n$ de base $B$ et $B'$.
		
		Alors $L$ = $L'$ si et seulement si $\exists U \in M_n(\mathbb{Z})$ tel que $B'=BU$ et $U \in GL_n(\mathbb{Z})$
	\end{proposition}
	
	\begin{definition}
		La taille d'un réseau $L(B)$ est $det(B)$ et est noté $\|L\|$. La taille d'un réseau est indépendant de la base choisie.
	\end{definition} 
	
	En substituant $\mathbb{Z}$ par un anneau de polynômes, on obtient ce qu'on appelle les réseaux polynomiaux.
	
	\begin{example}
		
	\end{example}
	\begin{definition}[\cite{Wallet}]
		On appelle \textbf{minimum d'un réseau} $\mathcal{L}$ la quantité
		
		$$\lambda_1 = min\{r>0 : |\mathcal{B}(r)\cap\mathcal{L}|>1\} \in \mathbb{R_+}$$
	\end{definition}
	
	
	On a vu que différentes bases peuvent être associées à un même réseau. Existe-t-il une notion de "bonne base" ? Nous verrons qu'une base idéale est celle qui est la plus orthogonale possible. Cette question est en lien avec des problèmes ouverts majeurs, tels que leShortest Vector Problem (SVP) et le closest vector problem 
	 (CVP).
	\begin{example}
		Les bases $B = \left(\begin{array}{rr}1 & 0 \\0 & 1\end{array}\right)$ et $B'=\left(\begin{array}{rr}1 & -1 \\-2 & 1\end{array}\right)$ engendrent le même réseau.
		
		Ici $U = B'^{-1} = \left(\begin{array}{rr}-1 & -1 \\-2 & -1\end{array}\right)$ et on voit que $det(U)=-1$. Donc $B$ et $B'$ engendrent le même réseau $\mathcal{L}$.
		
		On a $\|L\| = 1$ et $\lambda_1( \mathcal{L}) = 1$
	\end{example}
	
	\subsection{Quelques problèmes algorithmiques liés aux réseaux euclidiens}
	
	Le calcul du plus court vecteur dans un réseau est un problème difficile.
	
	Considérons le problème suivant :
	
	\begin{itemize}
		\item \textbf{Shortest Vector Problem (SVP)} : Étant donnée une base $B$ d’un réseau $L$, trouver un vecteur $v \neq 0$ tel que $\|v\| = \lambda_1(L)$. Ce problème est \textbf{NP-complet} (Ajtai).
	\end{itemize}
	
	On s’intéresse souvent à une version approximative plus accessible :
	
	\begin{itemize}
		\item \textbf{SVP$_\gamma$}, où $\gamma > 0$ : Étant donnée une base $B$ du réseau $L$, trouver un vecteur $v \neq 0$ tel que $\|v\| \leq \gamma \cdot \lambda_1(L)$.
	\end{itemize}
	
	L’état des connaissances actuelles est le suivant :
	
	\begin{itemize}
		\item Pour $\gamma = O(1)$, le problème reste \textbf{NP-complet}.
		\item Pour $\gamma = \text{poly}(n)$, il existe des algorithmes en \textbf{temps exponentiel}.
		\item Pour $\gamma = 2^{O(n)}$, l’algorithme \textbf{LLL} permet de le résoudre en \textbf{temps polynomial}.
	\end{itemize}
	
	Un autre problème important concerne la recherche de vecteurs proches dans un réseau.
	
	\begin{itemize}
		\item \textbf{Closest Vector Problem (CVP)} : Étant donnés une cible $t \in \mathbb{R}^m$ et un réseau $L(B)$, trouver un vecteur $v \in L$ tel que 
		\[
		\|t - v\| = d(t, L) := \min \{ \|t - v\| \mid v \in L \}.
		\]
	\end{itemize}
	
	De même, on peut considérer une version approximative :
	
	\begin{itemize}
		\item \textbf{CVP$_\gamma$}, où $\gamma > 0$ : Trouver un vecteur $v \in L$ tel que 
		\[
		\|t - v\| \leq \gamma \cdot d(t, L).
		\]
	\end{itemize}
	
	Le problème CVP est en général difficile pour un réseau arbitraire. Cependant, pour certaines familles spécifiques de réseaux, comme $\mathbb{Z}^n$, des algorithmes en temps polynomial sont connus. La qualité de la base choisie joue un rôle crucial dans la résolution du problème.
		
	\section{Réduction de réseaux euclidiens et polynomiaux}
	
	
	La réduction de réseaux est un outil fondamental en cryptographie et, plus généralement, en calcul formel. La réduction d'un réseau consiste à modifier une base quelconque de ce réseau en une base presque orthogonale. L'intérêt est de trouver de ``bonnes'' bases afin de résoudre divers problèmes.
	
	Nous avons deux résultats distincts concernant la réduction de réseaux euclidiens et polynomiaux :
	
	\begin{itemize}
		\item La réduction de réseaux sur $\mathbb{F}[X]$ s'effectue en temps polynomial.
		\item La réduction de réseaux sur $\mathbb{Z}$ est $\mathsf{NP}$-difficile.
	\end{itemize}	
	
	\subsection{Réduction de réseaux euclidiens}
	
	\subsubsection{Généralités}
	
	\begin{definition}{\cite{MCA}}
		Soit $f_1, \cdots, f_n \in \mathbb{R}^n$ une base et $f_1^*, \cdots, f_n^*\in\mathbb{R}^n$ sa base de Gram-Schmidt associée.
		
		On dit que $(f_1, \cdots, f_n)$ est \textbf{réduite} si $\|f_i^*\|^2 \leq 2 \| f_{i+1}^*\|^2$ pour $1 \leq i \leq n$ 
	\end{definition}
	
	L'algorithme LLL (proposé par Lenstra et al. en YYYY) repose sur un principe simple : il calcule la meilleure approximation entière de la décomposition de Gram-Schmidt et réduit la base en réorganisant les vecteurs si nécessaire. Cet algorithme est décrit dans \cite{MCA}.
	
	Le code complet de l'algorithme est présenté dans l'annexe \ref{sec:annexe_code}.
	\begin{algorithm}
		\caption{BasisReduction (LLL) \cite{MCA}}
		\label{algo:LLL_MCA} 
		\ref{algo:LLL_MCA}
			
		\begin{algorithmic}[1]
			\State \textbf{Entrée :} Une base $B=(f_1, \cdots, f_n)$
			\State \textbf{Sortie :} Une base réduite $G=(g_1, \cdots, g_n)$ de $B$
			\For{$i = 1$ \textbf{à} $n$}
				\State $g_i \coloneqq f_i$
			\EndFor
			
			\State $(B^*, U)\coloneqq$ GSO$(B)$
			
			\While{$i \leq n$}
			\For{$j=i-1, i-2$ \textbf{à} $1$}
			\State $g_i:=gi-\lfloor \mu_{ij} \rceil g_j$
			\State Mettre à jour $(B^*, U)\coloneqq$ GSO$(B)$
			\EndFor
			\If{$i>1$ et $\|f_i^*\|^2 > 2 \| f_{i+1}^*\|^2$}
			\State échanger $g_{i-1}$ et $g_i$
			\State Mettre à jour $(B^*, U)\coloneqq$ GSO$(B)$
			\State $i \coloneqq i-1$ 
			\Else 
			\State $i \coloneqq i+1$
			\EndIf
			\EndWhile
			\State Retourner $G=(g_1, \cdots, g_n)$
		\end{algorithmic}
	\end{algorithm}
	
	\subsection{Réduction de réseaux polynomiaux}
	Cette section sur la réduction des réseaux polynomiaux est fortement inspirée de \cite{CCLebreton}.  
	
	La réduction des réseaux polynomiaux est un outil essentiel. Par exemple, elle peut être utilisée pour le décodage des codes de Reed-Solomon. Dans cette partie, nous présenterons les idées et les outils permettant de réduire les réseaux polynomiaux en temps polynomial, en nous appuyant sur les meilleurs exposants connus à ce jour.  
	
	Lorsque l'on travaille avec des matrices à coefficients dans un corps fini \(\mathbb{F}\), de nombreuses opérations ont des complexités équivalentes : la multiplication de matrices, l'inversion d'une matrice, le calcul du déterminant ou encore la résolution d'un système linéaire. Mais qu'en est-il lorsque les matrices ont leurs coefficients dans \(\mathbb{F}[X]\) ?  
	
	Dans ce cas, le calcul du déterminant est équivalent à la multiplication de matrices. D'autres opérations, comme l'ordonnancement des bases et la réduction de colonnes, restent également de complexité comparable. En revanche, l'inversion d'une matrice ne l'est plus, en raison de la taille de la sortie.
	
	
	La base d'ordre est un concept imortant en travaillant avec les matrices polynomiales pour réduire beaucoup de problemes à la multiplicatiob
	
	Soit F un corps fii, F X <=d les polyonme sur F de degré <= d et F x m n les matrices m n avex coeff polynomiau
	
	slide 6
	
	
	Soit $F \in \mathbb{F}[X]^{m \times n}$. On définit $(F, \sigma):= \{v \in \mathbb{F}^{1 \times m}$ tel que $ vF = 0$ mod $x^\sigma\}$
	
	\begin{proposition}
		$(F, \sigma)$ est un $\mathbb{F}[X]$ module de dimension $m$.
	\end{proposition}

	preuve : Il faut montrer que c'est un sous-groupe additif stable pour la multiplication?
	
	
	\begin{definition}
		Une base d'ordre P de F $\sigma$
		 est une base du F X module de F $\sigma$ de degré minimale
	\end{definition}
	
	mais qsuelle est la notionn de dégré, quelle ets la notion d'ordre et de miniaité ?
	
	
	\begin{definition}[\cite{clef_unique_0}]
		Soit $\mathbb{K}$ un corps et $M \in \mathbb{K}[x]_{1 \times n}$. On définit le \textbf{degré de ligne} du vecteur ligne M par : 
		$$rdeg(M)=max(deg(m_i))_{i\in\{1, \cdots, n\}}$$
	\end{definition}
	
	\begin{definition}[\cite{clef_unique_0}]
		Soit $\mathbb{K}$ un corps et $M \in \mathbb{K}[x]_{n \times n}$. On définit le \textbf{degré de ligne} de la matrice M par :
		$$rdeg(M)=max(rdeg(ligne_i))_{i\in\{1, \cdots, n\}}$$
	\end{definition}
	
	\begin{example}
		Soit $M =
		\left(\begin{array}{rrrr}
			1 & 0 & 1 & 1 \\
			x & 1 & x + 1 & 0 \\
			1 & x^{3} + x^{2} & x & 0 \\
			x^{2} & 0 & x^{4} + x^{3} & 0
		\end{array}\right) \in \mathbb{F}_2[x]
		$
		alors $row\_degree(M)=\left(\begin{array}{rrrr}
			0 & 1 & 3 & 4
		\end{array}\right)\in \mathbb{Z}^4$
		
	\end{example}
	
	Mais on a un problème
	
	\begin{definition}
		Soit $s \in \mathbb{Z}^n$. On définit le degré de ligne décalé du vecteur ligne M par :
		$$rdeg_s(M)=max(deg(m_i)+s_i\\)$$
	\end{definition}
	degré décalé d'une matrice
	
	exemple
	
	notation $x^s$
	
	remarque 2
	
	transitivité slide 12
	
	On va définir un ordre sur les row degree
	
	
	
	matrice row reduced def
	
	
	pour parler d'un minimum faudrait un ordre total
	
	algo DAC
	 je cite \cite{clef_unique_2} et puis \cite{clef_unique_3} et finalement \cite{clef_unique_4}
	
	\chapter{Adaptation de la réduction de réseaux polynomiaux au cas entier}
	
	\chapter{Réseaux définis par relations plutôt que par générateurs}
	\chapter*{Conclusion et perspectives}
	
	\appendix
	
	\titleformat
	{\chapter} % command
	[display] % shape
	{\Large} % format
	{%
		\rule{\widthof{ANNEXE}}{0.6pt}\vspace{-0.3cm}\\
		\makebox[0pt][l]{ANNEXE \Huge\textbf{\thechapter}}\vspace{-0.6cm} \\
		\rule{\widthof{ANNEXE \Huge\textbf{\thechapter}}}{0.6pt}
	} % label
	{1ex} % sep
	{\huge\textit } 
	[
	] % after-code
	
	
	%rajouter des corps ?
	\chapter{Rappels d'algèbre : Groupes, Anneaux et Modules}

	\section{Définition et propriétés des groupes}
	
	
	subsection Généralités
	
	definition
	
	exemple contre exemple
	
	
	\section{Définition et propriétés des anneaux}
	
	\subsection{Généralités}
	
	\begin{definition}
		Un \textbf{anneau} $(A, +, \cdot)$ est un ensemble : 
		\begin{itemize}
			\item muni d'une loi interne $+$ tel que $(M, +)$ est un groupe abélien.
			\item muni d'une loi interne $A \times A \rightarrow A, (a, b) \mapsto a \cdot b$ qu'on notera $ab$ vérifiant les trois propriétés suivantes :
			
			\begin{enumerate}
				\item $\cdot$ distributif sur $+$ :
					\begin{itemize}
						\item $a(b + c) = ab + ac$
						\item $(b+c)a = ba+ca$
					\end{itemize}
				
				
				\item $\cdot$ est associatif
				
				\item $\cdot$ a un élément neutre
			\end{enumerate}
		\end{itemize}
	\end{definition}

	\begin{definition}
		Un anneau $A$ est \textbf{commutatif} si $ab = ba$ pour tout $a$, $b \in A$
	\end{definition}
	
	\begin{definition}
		Un anneau $A$ est \textbf{intègre} si $ab=0 \Rightarrow a = 0$ ou $b = 0$ pour tout $a$, $b \in A$
	\end{definition}

	\begin{definition}
		Soit $A$ un anneau. Un \textbf{idéal} $I$ de $A$ est un sous-ensemble de $A$ tel que
		\begin{enumerate}
			\item $I$ est un sous-groupe additif de $(A, +)$
			\item pour tout $a \in A$, $x \in I$, on a $ax \in I$ et $xa \in I$
		\end{enumerate}
	\end{definition}

	def anneau euclidien
	
	\begin{definition}
		Un idéal est dit \textbf{principal} s'il est engendré par un élément. Un anneau A est dit \textbf{principal} si tous ses anneaux sont principaux.
	\end{definition} 
	
		

	def Anneau noetherien
	
	prop anneau noetherien
	
	
	def anneau factoriek
	
	\begin{theoreme}
		Tout anneau principal est factoriel.
	\end{theoreme}
	
	
	\begin{theoreme}
		Tout anneau euclidien est principal
	\end{theoreme}
	
	\begin{theoreme}
		Tout anneau principal est noetherien
	\end{theoreme}
	
	
	\begin{theoreme}
		ctout anneau commutatif intègre et principal est foactoriel
	\end{theoreme}
	
	
	

	
	
	proposition des implications
	schéma d'inclusion récap pq pas.
	
	\subsection{Anneaux de polynômes}
	
	Dans ce mémoire, les anneaux de polynômes constituent un type d'anneaux particulièrement utile.
	
	\begin{theoreme}
	Si $A$ est un anneau factoriel, alors $A[X]$ est un anneau factoriel.
	\end{theoreme}

	\begin{remark}
		Il s'en suit que $A[X, Y]$ est factoriel et ainsi de suite.
	\end{remark}

	\begin{theoreme}
		Si $A$ est un anneau Noethérien, alors $A[X]$ est un anneau Noethérien.
	\end{theoreme}
	
	\begin{remark}
		Il s'en suit que $A[X, Y]$ est Noethérien et ainsi de suite.
	\end{remark}

	\begin{theoreme}
		Si $A$ est un anneau factoriel, alors $A[X]$ est un anneau intègre.
	\end{theoreme}
	
	\begin{remark}
		Il s'en suit que $A[X, Y]$ est intègre et ainsi de suite.
	\end{remark}

	Si K corps K X principal je crois
	
	\begin{proposition}
		$K[X]$ est principal.
	\end{proposition}

	\begin{proposition}
		$K[X, Y]$ n'est pas principal.
	\end{proposition}
	
	\section{Définition et propriétés des modules} %pq pas exmeple ?
	
	\subsection{Généralités}
	Cette partie se base sur \cite{ring_modules} et \cite{harari_modules}.
	
	La notion de module est la généralisation naturelle de celle d'espace vectoriel. Dans toute la suite $A$ désigne un anneau commutatif.
	
	\begin{definition}
		Un \textbf{$A$-module} $(M, +, \cdot)$ est un ensemble :
		
		\begin{itemize}
			\item muni d'une loi interne $+$ tel que $(M, +)$ est un groupe abélien.
			\item muni d'une loi externe $A \times M \rightarrow M, (a, m) \mapsto a \cdot m$, qu'on notera $am$ vérifiant les quatre propriétés suivantes :
			\begin{enumerate}
				\item Distributivité : $\alpha(m + m') = \alpha m + \alpha m'$
				\item Distributivité : $(\alpha+\beta)m = \alpha m + \beta m$
				\item Associativité : $(\alpha \beta)m = \alpha(\beta m)$
				\item Neutre : $1m = m$
				
				pour tout $\alpha$, $\beta \in A$ et $m$, $m' \in M$
			\end{enumerate}
		\end{itemize}
	\end{definition}
	
	\begin{remark}
		La définition d'un module peut être vue comme une généralisation de celle d'un espace vectoriel. En effet, un module est un espace vectoriel dans lequel l'hypothèse sur $A$ est affaiblie : au lieu d'être un corps, $A$ est un anneau, ici commutatif, on ne se soucie donc pas de la distinction entre modules à gauche ou à droite.
	\end{remark}

	%plus on affaibli les hypothèses plus l'objet est général et donc plus difficile à étudier
	
	\begin{example}
		Soit $n \in \mathbb{N}$, $\mathbb{Z}/n \mathbb{Z}$ est un $\mathbb{Z}$-module.
	\end{example}

	\begin{definition}
		Soit $M$ un $A$-module. Un \textbf{sous-module} $N$ de $M$ est un sous-groupe de $(M, +)$ stable pour la multiplication externe par tout élément de $A$.
	\end{definition}

	% contre exemple au fait d'etre stable pour la multiplication externe ?
	
	%exemple et contre-exemple de modules
	
	\begin{definition}
		Un $A$-module $M$ est dit de \textbf{type fini} s'il existe une partie finie $S$ de $M$ tel que $M$ soit engendré par $S$.
	\end{definition}

	\begin{definition}
		Un $A$-module $M$ est dit \textbf{libre} s'il admet une base i.e. si il admet une famille $(x_i)_{i \in I}$ tel que pour tout $x \in M$, $x$ s'écrit de manière unique comme $\sum_{i \in I} \alpha_i x_i$
	\end{definition}

	\begin{proposition}
		Si $M$ est libre et de type fini, alors il admet une base finie. On dit dans ce cas que $M$ est libre de type fini, et toutes les bases de $M$ ont le même cardinal, qu'on appelle rang de $M$.
	\end{proposition}

	\begin{example}
		$\mathbb{Z}/n \mathbb{Z}$ est un $\mathbb{Z}$-module de type fini mais pas libre pour $n \neq 0$.
	\end{example}
		
	\subsection{Modules sur un anneau principal}
	Dans cette partie, $A$ désigne un anneau principal.
	
	\begin{theoreme}
	Si M est un module libre sur un anneau principal A, tout sous-module de M est libre et de rang inférieur ou égal à celui de M.
	\end{theoreme}


	\begin{example}
		Dans le $\mathbb{Z}$-module libre $\mathbb{Z}^2$, le sous-module des couples $(a, b)$ tels que $a \equiv b$ mod $10$ est libre de base $((1,1), (0, 10))$ donc de rang $2$ (comme le module lui-même).
	\end{example}
	
	
	
	\chapter{Rappels d'algèbre linéaire}
	
	Dans ce mémoire, les matrices sont considérées comme des matrices dont les éléments sont des vecteurs ligne.
	
	% demander à Romain si je met GS ici ou ailleurs
	
	\section{Orthonormalisation de Gram-Schmidt (GSO)}
	
	On se considère dans un espace euclidien $\mathbb{R}^n$. Soit \( B = (b_i)_{1 \leq i \leq n} \) une base de \( \mathbb{R}^n \), on peut construire la \textbf{base de Gram-Schmidt (GS)} associée à cette base, que l'on notera \( B^* = (b^*_i)_{1 \leq i \leq n} \) par le procédé récursif suivant :
	
	
	$$b_1^* := b_1$$
	
	
	$$b_i^* := b_i - \sum_{j=1}^{i-1} \mu_{i,j} b_j^*, \quad \text{avec } \mu_{i,j} = \frac{\langle b_i, b_j^* \rangle}{\| b_j^* \|^2}$$
	
	où \( \mu_{i,j} \) est appelé \textbf{coefficient de Gram-Schmidt} associé au vecteur \( b_i \). \\
	
	Notons \( B^* \) la matrice des \( b_i^* \), et \( U = \left(\begin{array}{rrrr}
		1 & 0 & \cdots & 0\\
		\mu_{2,1} & \ddots & \ddots & \vdots\\
		\vdots & \ddots & \ddots & 0\\
		\mu_{n,1} & \cdots & \mu_{n,n-1} & 1
	\end{array}\right)\). 

Alors, on a la relation : $B = UB^*$
	
	 \begin{remark}
	 	\begin{itemize}
	 		\item  $U$ est triangulaire avec un déterminant égal à $1$.
	 		\item Contrairement à ce qui est souvent pratiqué dans la littérature, nous n'effectuons pas de normalisation, car cela introduirait des racines, ce qui pourrait entraîner un passage des coefficients \( \mathbb{Q} \) à des coefficients \( \mathbb{R} \setminus \mathbb{Q} \). De plus, la normalisation impose un volume égal à 1, ce qui entraîne une perte de l'information volumétrique du réseau associé.
	 		
	 	\end{itemize}
 	\end{remark}
	
	
	\begin{proposition}
		 Soit $B = (b_i)_{1 \leq i \leq n}$ une base de $\mathbb{R}^n$, et $B^* = (b^*_i)_{1 \leq i \leq n}$ sa base de Gram-Schmidt associée.
		 
		 On a $det(B)=det(B^*) = \prod \| b_i^*\|$
	\end{proposition}	

	\begin{proof}
		Nous utilisons le fait que \( U \) est une matrice triangulaire dont les coefficients diagonaux sont égaux à 1, et que \( B^* \) est une base de vecteurs orthogonaux. Ainsi, 
		\[\det(B) = \det(UB^*) = \det(U)\det(B^*) = \det(B^*) = \prod \| b_i^*\|.\]
	\end{proof}
	
	
	%	\item Propriété du plus petit vecteur** : \\lambda_1(\mathcal{L}) \geq \min \| b_i^* \| \)
	
	
	\begin{algorithm}
		\caption{GramSchmidt (GSO)}
		\label{alg:GSO} 
		\ref{alg:GSO}
		
		\begin{algorithmic}[1]
			\State \textbf{Entrée :} Une base $B=(f_1, \cdots, f_n)$
			\State \textbf{Sortie :} La base de Gram-Schmidt associée $B^*=(b^*_1, \cdots, b^*_n)$ et $U$ la matrice des coefficients de GS
			
			\For{$k = 1$ \textbf{à} $n$}
			\State $b_k^* \coloneqq b_k$
			\For{$j = 1$ \textbf{à} $k$}
			\State $U_{kj}\coloneqq \frac{(b_k*b_j^*)}{\|b_j^*\|^2}$
			\State $b_k^* \coloneqq U_{kj} b_j^*$
			\EndFor
			\EndFor
		\end{algorithmic}
	\end{algorithm}
	\begin{theoreme}
		Le coût de l'algorithme \nameref{alg:GSO} est de \( O(n^3) \) opérations arithmétiques dans \( \mathbb{Q} \).
	\end{theoreme}
	
	
	\chapter{Code SageMath}
	Le code source complet est également disponible sur \url{https://github.com/toncompte/tonrepo}.
	\label{sec:annexe_code}
	Voici le code complet de l'algorithme implémenté :
	
	\begin{lstlisting}
	def Gram_Schmidt(B):
	n = B.nrows()
	
	B_star = Matrix(QQ, n, n)
	U = identity_matrix(QQ, n)
	
	B_star[0] = B[0]
	
	for k in range(1, n):
	B_star[k] = B[k]
	for j in range(k):g
	U[k, j] = (B[k] * B_star[j]) / (B_star[j] * B_star[j])
	B_star[k] -= U[k, j]*B_star[j]
	return U, B_star
	
	\end{lstlisting}
	
	
	

	
	\printbibliography
	
\end{document}
