\documentclass[a4paper,12pt]{report}  % Déclare un document de type "report" (idéal pour les rapports longs)
% a4paper : Définit le format du papier en A4 (210 × 297 mm)
% 12pt : Définit la taille de police principale à 12 points (standard pour les rapports)

\usepackage[utf8]{inputenc} % Permet d'écrire directement des accents et caractères spéciaux avec l'encodage UTF-8

\usepackage[T1]{fontenc}  % Définit l'encodage des polices pour une meilleure gestion des caractères
% T1 : Active un encodage qui améliore l'affichage des accents (é, à, ç...) 
%      et la césure correcte des mots en français
% Recommandé pour les documents en français

\usepackage{amsmath, amsthm, amssymb}   % amsmath :  Pour des équations et des                                         notations mathématiques avancées.
% amsthm : Pour formater des théorèmes, définitions, propositions, etc.
% amssymb : Pour ajouter des symboles mathématiques supplémentaires

% Définition d'un style pour les définitions
\newtheoremstyle{definitionstyle}  % Nom du style
{10pt}                           % Espace avant
{10pt}                           % Espace après
{\normalfont}                    % Police normale
{}                               % Retrait
{\bfseries}                      % Titre en gras
{.}                              % Ponctuation après le titre
{1em}                            % Espace après le titre (ajout d'un petit                                    espace)
{}                               % Personnalisation du titre

\theoremstyle{definitionstyle}
\newtheorem{definition}{Définition}[chapter] % Numérotation par chapitres

% Définition d'un style pour les exemples
\newtheoremstyle{examplestyle}  % Nom du style
{10pt}                        % Espace avant
{10pt}                        % Espace après
{\normalfont}                 % Police normale
{}                            % Retrait
{\itshape}                    % Titre en italique
{.}                           % Ponctuation après le titre
{1em}                         % Espace après le titre
{}                            % Personnalisation du titre

\theoremstyle{examplestyle}
\newtheorem{example}{Exemple}[chapter] % Numérotation par section

\newtheoremstyle{remarkstyle}  
{10pt}                        % Espace avant
{10pt}                        % Espace après
{\normalfont}                 % Police normale pour le texte
{}                            % Retrait
{\itshape}                    % Titre en italique
{.}                           % Ponctuation après le titre
{1em}                         % Espace après le titre
{}                            % Personnalisation du titre

\theoremstyle{remarkstyle}
\newtheorem{remark}{Remarque}[chapter] % Numérotation par chapitres


\usepackage[a4paper, left=2.5cm, right=2.5cm, top=2.5cm, bottom=2.5cm]{geometry} % Définition des marges du document (A4 avec 2.5 cm sur chaque côté)

\usepackage[backend=biber,style=authoryear]{biblatex} % Gestion de la bibliographie
\addbibresource{references.bib} % On inclue le fichier BibTeX references

\usepackage{fancyhdr}     % En-têtes et pieds de page personnalisés

\pagestyle{fancy}         % modification flexible de l'en-tête et du pied de                             page  
\fancyhf{}                % vide l'en-tête et le pied de page
\fancyhead[L]{\leftmark}  % Section ou chapitre courant
\fancyfoot[C]{\thepage}   % Numéro de page au centre

\usepackage{hyperref}
% Ajouter des liens hypertextes dans le document                                 (liens internes, externes, etc.), mettre les liens                             en surbrillance

%%%%%%%%%%%%%%%%%%%%%%%%%%%%%%%%%%%%%%%%%%%%%%%%%%%%
%%%%%%%%%%%%%%%%%%%%%%%%%%%%%%%%%%%%%%%%%%%%%%%%%%%%
%%%%%%%%%%%%%%%%%%%%%%%%%%%%%%%%%%%%%%%%%%%%%%%%%%%%
\usepackage{csquotes}
\usepackage[english,french]{babel} % Langue du document
\usepackage{graphicx}       % Inclusion d'images
\usepackage{float}
\usepackage{color}
\usepackage{comment}        % pour commenter plusieur ligne



\begin{document}
	
	% En-tête
	\begin{titlepage}
		\begin{center}
			% Logos
			\begin{figure}[t]
				\centering
				\includegraphics[width=0.35\textwidth]{logo/logo_LIRMM.jpg} 
			\end{figure}
			\begin{figure}[t]
				\centering
				\includegraphics[width=0.2\textwidth]{logo/logo_univ_mpt.png} 
			\end{figure}
			
			% Titre
			\vspace{2cm}
			\Huge{\textbf{Rapport de stage : Réduction de réseaux - adaptations d'idées provenant du cas polynomiale au cas entier.}} \\
			\vspace{0.5cm}
			\large{\textit{HAI001I : Stage académique}} \\
			
			% Espacement
			\vspace{2cm}
			
			% Auteurs
			\large{
				\textbf{Lucas Noirot} \\
				(\texttt{lucas.noirot@etu.umontpellier.fr})
			}
			
			% Espacement
			\vspace{1.5cm}
			
			% Encadrant
			\large{
				\textbf{Encadrant :} \textbf{Romain Lebreton} \\
				(\texttt{romain.lebreton@lirmm.fr})
			}
			
			% Espacement
			\vspace{2cm}
			
			% Date
			\normalsize{
				\textbf{Date :}  17 février - 26 juin 2025
			}
		\end{center}
	\end{titlepage}
	
	\chapter*{Remerciements}
	% Table des matières
	\tableofcontents
	
	\phantomsection
	\chapter*{Notations}
	\addcontentsline{toc}{chapter}{Notations}  % Ajoute au sommaire si nécessaire
	
	$\mathbb{Z}$ désigne l'ensemble des entiers relatifs.
	
	$\mathbb{R}$ désigne l'ensemble des réels.	
	
	Les réseaux seront désignés par une lettre majuscule calligraphiée, telle que $\mathcal{L}$.
	
	\chapter*{Guide de lecture}

	
	
	
	
	
	
	\chapter*{Introduction}
	
	L'objectif de mon stage est d'étudier les réseaux euclidiens et polynomiaux, ainsi que d'analyser les techniques de réduction de réseau afin d'adapter celles du cas polynomial au cas entier.
	
	La réduction de réseau polynomiaux est un outil important en cryptographie 
	
	\chapter{Réseaux euclidiens et polynomiaux}
	
	\section{Généralités sur les réseaux}
	\subsection{Définitions et exemples}
	Nous considérons un espace euclidien, c'est-à-dire un espace vectoriel réel de dimension finie muni d'un produit scalaire, noté $\langle f, g \rangle$. Ici, nous utiliserons le produit scalaire usuel, défini par $\langle f, g \rangle = f \cdot g^T$ lequel induit la norme-2, donnée par $\|f\|_2=(f \cdot f^T)^{1/2}$.
	
	\begin{definition}%%%%%%ùfinir cette def
		Soit $n \in \mathbb{N}$. Z module reseau "euclidien"?
	\end{definition}

	\begin{definition}[\cite{clef_unique_1}]
		Un \textbf{réseau euclidien} est un sous-groupe discret de $\mathbb{R}^n$
	\end{definition}

	%pour etre discret faut une norme

	\begin{definition}[\cite{clef_unique_2}]
		Un réseau euclidien $(\Lambda, q)$ est un $\mathbb{Z}$-module libre $\Lambda$ de rang fini avec une forme quadratique définie positive $q$ sur $\Lambda \otimes_\mathbb{Z} \mathbb{R}$ 
	\end{definition}
	
	%parler des reseaux polynomiauxEn remplacant Z par anneau de polynome, on a des reseaux polynomiaux.
	
	%Ces définitions sont équivalentes??
	
	\begin{example}
		
	\end{example}
	def de la base d'un reseau
	
	def d'un plus court vecteur
	
	\begin{definition}[\cite{clef_unique_1}]
		On appelle minimum d'un réseau $\mathcal{L}$ la quantité
		
		$$\lambda_1 = min\{r>0 : |\mathcal{B}(r)\cap\mathcal{L}|>1\} \in \mathbb{R_+}$$
	\end{definition}
	
	
	problematique -> differentes bases -> des bases avec de cours vecteurs,
	
	
	
	
	parler des problemes SVP
	 probleme NP -difficile
	
	\subsection{Quelques problèmes algorithmiques liés aux réseaux euclidiens}
	
	Le calcul du plus court vecteur dans un réseau est un problème difficile.
	
	Considérons le problème suivant :
	
	\begin{itemize}
		\item \textbf{Shortest Vector Problem (SVP)} : Étant donnée une base $B$ d’un réseau $L$, trouver un vecteur $v \neq 0$ tel que $\|v\| = \lambda_1(L)$. Ce problème est \textbf{NP-complet} (Ajtai).
	\end{itemize}
	
	On s’intéresse souvent à une version approximative plus accessible :
	
	\begin{itemize}
		\item \textbf{SVP$_\gamma$}, où $\gamma > 0$ : Étant donnée une base $B$ du réseau $L$, trouver un vecteur $v \neq 0$ tel que $\|v\| \leq \gamma \cdot \lambda_1(L)$.
	\end{itemize}
	
	L’état des connaissances actuelles est le suivant :
	
	\begin{itemize}
		\item Pour $\gamma = O(1)$, le problème reste \textbf{NP-complet}.
		\item Pour $\gamma = \text{poly}(n)$, il existe des algorithmes en \textbf{temps exponentiel}.
		\item Pour $\gamma = 2^{O(n)}$, l’algorithme \textbf{LLL} permet de le résoudre en \textbf{temps polynomial}.
	\end{itemize}
	
	Un autre problème important concerne la recherche de vecteurs proches dans un réseau.
	
	\begin{itemize}
		\item \textbf{Closest Vector Problem (CVP)} : Étant donnés une cible $t \in \mathbb{R}^m$ et un réseau $L(B)$, trouver un vecteur $v \in L$ tel que 
		\[
		\|t - v\| = d(t, L) := \min \{ \|t - v\| \mid v \in L \}.
		\]
	\end{itemize}
	
	De même, on peut considérer une version approximative :
	
	\begin{itemize}
		\item \textbf{CVP$_\gamma$}, où $\gamma > 0$ : Trouver un vecteur $v \in L$ tel que 
		\[
		\|t - v\| \leq \gamma \cdot d(t, L).
		\]
	\end{itemize}
	
	Le problème CVP est en général difficile pour un réseau arbitraire. Cependant, pour certaines familles spécifiques de réseaux, comme $\mathbb{Z}^n$, des algorithmes en temps polynomial sont connus. La qualité de la base choisie joue un rôle crucial dans la résolution du problème.
	
	%%%%%%%%%%%%%%%%%%%%%%%%%%%
	def Orthogonalisation de Gram-Scmidt
	exemple
	parler de la complexité de Gram Schmidt
	%%%%%%%%%%%%%%%%%%%%%%%%%%%
	
	
	\section{Réduction de réseaux euclidiens et polynomiaux}
	
	\subsection{Réduction de réseaux euclidiens}
	
	\subsection{Réduction de réseaux polynomiaux}
	
	reseau reduit chap 16
	
	\begin{definition}[\cite{clef_unique_0}]
		Soit $\mathbb{K}$ un corps et $M \in \mathbb{K}[x]_{1 \times n}$. On définit le \textbf{degré de ligne} du vecteur ligne M par : 
		$$rdeg(M)=max(deg(m_i))_{i\in\{1, \cdots, n\}}$$
	\end{definition}
	
	\begin{definition}[\cite{clef_unique_0}]
		Soit $\mathbb{K}$ un corps et $M \in \mathbb{K}[x]_{n \times n}$. On définit le \textbf{degré de ligne} de la matrice M par :
		$$rdeg(M)=max(rdeg(ligne_i))_{i\in\{1, \cdots, n\}}$$
	\end{definition}
	
	\begin{example}
		Soit $M =
		\left(\begin{array}{rrrr}
			1 & 0 & 1 & 1 \\
			x & 1 & x + 1 & 0 \\
			1 & x^{3} + x^{2} & x & 0 \\
			x^{2} & 0 & x^{4} + x^{3} & 0
		\end{array}\right) \in \mathbb{F}_2[x]
		$
		alors $row\_degree(M)=\left(\begin{array}{rrrr}
			0 & 1 & 3 & 4
		\end{array}\right)\in \mathbb{Z}^4$
		
	\end{example}
	
	\begin{definition}
		Soit $s \in \mathbb{Z}^n$. On définit le degré de ligne décalé du vecteur ligne M par :
		$$rdeg_s(M)=max(deg(m_i)+s_i\\)$$
	\end{definition}

	Je cite \cite{clef_unique_1}  je cite \cite{clef_unique_2} et puis \cite{clef_unique_3} et finalement \cite{clef_unique_4}
	
	\chapter{Adaptation de la réduction de réseaux polynomiaux au cas entier}
	
	\chapter{Réseaux définis par relations plutôt que par générateurs}
	\chapter*{Conclusion et perspectives}
	
	\appendix
	%rajouter des corps ?
	\chapter{Rappels d'algèbre : Groupes, Anneaux et Modules}

	\section{Définition et propriétés des groupes}
	
	\section{Définition et propriétés des anneaux}
	\section{Définition et propriétés des modules} %pq pas exmeple ?
	
	\subsection{Généralités}
	
	La notion de module est la généralisation naturelle de celle d'espace vectoriel. Dans toute la suite $A$ désigne un anneau commutatif.
	
	\begin{definition}
		Un \textbf{$A$-module} $(M, +, \cdot)$ est un ensemble :
		
		\begin{itemize}
			\item muni d'une loi interne $+$ tel que $(M, +)$ est un groupe abélien.
			\item muni d'une loi externe $A \times M \rightarrow M, (a, m) \mapsto a \cdot m$, qu'on notera $am$ vérifiant les quatre propriétés suivantes :
			\begin{enumerate}
				\item Distributivité : $\alpha(m + m') = \alpha m + \alpha m'$
				\item Distributivité : $(\alpha+b)m = \alpha m + \beta m$
				\item Associativité : $(\alpha \beta)m = \alpha(\beta m)$
				\item Neutre : $1m = m$
				
				pour tout $\alpha$, $\beta \in Ala 3$ et $m$, $m' \in M$
			\end{enumerate}
		\end{itemize}
	\end{definition}
	
	\begin{remark}
		La définition d'un module peut être vue comme une généralisation de celle d'un espace vectoriel. En effet, un module est un espace vectoriel dans lequel l'hypothèse sur $A$ est affaiblie : au lieu d'être un corps, $A$ est un anneau, ici commutatif, on ne se soucie donc pas de la distinction entre modules à gauche ou à droite.
	\end{remark}
	
	\begin{definition}
		Soit $M$ un $A$-module. Un sous-module $N$ de $M$ est un sous-groupe de (M, +) stable pour la multiplication externe par tout élément de A
	\end{definition}
	% contre exemple au fait d'etre stable pour la multiplication externe ?
	
	%exemple et contre-exemple de modules
	
	\begin{definition}
		type fini
	\end{definition}

	\begin{definition}
		libre
	\end{definition}


	proposition : libre et type fini => base finie
	
	exemples et contre-exemple
	
	%dire D	NS L'INTRO encore une fois que je vais donner des exemples et des contre exemple afin de mieux saisir les définitions
	
	%produit tensoriel
	
	\subsection{Modules sur un anneau principal}
	Dans cette partie, A désigne un anneau principal.
	
	

	
	\printbibliography
	
\end{document}
